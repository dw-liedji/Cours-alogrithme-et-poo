\section*{Exercice 1: \scriptsize{Questions de cours (1+0.5x5+1+1+1=6 points)}}
\begin{enumerate}
    \item Les extensions d'un fichier source C++ sont : .cpp et .h.
    \item Definitions des concepts:
          \begin{itemize}
              \item \textbf{Programmation orientée objet} c'est un paradigme de programmation inventé au
                    début des années 1960 consitant en la définition et l'interaction de briques logicielles appelées objets.
              \item \textbf{Un objet est une structure informatique regroupant} :
                    \begin{itemize}
                        \item des variables, caractérisant l’état de l’objet,
                        \item des fonctions, caractérisant le comportement de l’objet.
                    \end{itemize}
              \item \textbf{Classe}:  Une classe est un ensemble d'ojets de même type.
              \item \textbf{Polymorphisme}: c'est la capacité d'un l'objet à
                    posséder plusieurs formes.
              \item \textbf{Héritage}: C'est un mécanisme de la programmation orientée objet qui permet de créer des classes dites filles à partir des caractéristique et méthodes des classes existantes dites parentes.
          \end{itemize}
    \item Signification des mots clés:
          \begin{itemize}
              \item \textbf{Virtual}: signifie que toute fonction-membre de la classe de base doit être surchargée (c'est-à-dire redéfinie) dans une classe
                    dérivée.
              \item \textbf{Private}: signifie que les propriétés et méthodes d'une classe sont privées à cette classe et inaccéssible depuis l'exterieur.
          \end{itemize}
    \item La sortie du code est: \textbf{15}.
\end{enumerate}