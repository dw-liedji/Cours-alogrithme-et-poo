\section*{Problème:}

Réalisation d'une classe point permettant de manipuler un point d'un plan. Voici un exemple de fichier d'en-tête \textbf{Point.h} qui décrit un objet qui représente un point dans l'espace euclidien bidimensionnel :


\lstinputlisting[language=Caml, firstline=0, lastline=27]{code/2dex.h}
%\lstinputlisting[language=Caml, firstline=0, lastline=27]{code/2d.cpp}

\begin{enumerate}
    \item Ajoutez les propriétes privées de cette classe sachant que un point est défini par ses coordonnées x et y (des membres privés)

    \item Ajoutez un constructeur par défault et un constructeur paramétré.

    \item Ajoutez une fonction membre déplace effectuant une translation définie par ses deux arguments dx et dy (double)

    \item Ajoutez une fonction membre affiche se contentant d'afficher les coordonnées cartésiennes du point.

    \item Ajoutez une fonction membre saisir se contentant de saisir les coordonnées cartésiennes du point.

    \item Ajoutez une fonction membre distance effectuant calculant la distance entre deux point.

    \item Ajoutez une une fonction membre milieu donnant le milieu d'un segment.

    \item Ecrivez un petit programme d'essai (main) gérant la classe point.

\end{enumerate}
