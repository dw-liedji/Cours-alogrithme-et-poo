\section*{Exercice 1}

Réalisation d'une classe point permettant de manipuler un point d'un plan.:

\begin{enumerate}
      \item Ajoutez les propriétes privées de cette classe sachant que un point est défini par ses coordonnées x et y (des membres privés)
            \lstinputlisting[language=Caml, firstline=4, lastline=9]{code/2d.h}


      \item Ajoutez un constructeur par défault et un constructeur paramétré.
            \lstinputlisting[language=Caml, firstline=9, lastline=14]{code/2d.h}

      \item Ajoutez une fonction membre déplace effectuant une translation définie par ses deux arguments dx et dy (double)
            \lstinputlisting[language=Caml, firstline=59, lastline=68]{code/2d.cpp}

      \item Ajoutez une fonction membre affiche se contentant d'afficher les coordonnées cartésiennes du point.
            \lstinputlisting[language=Caml, firstline=78, lastline=85]{code/2d.cpp}

      \item Ajoutez une fonction membre saisir se contentant de saisir les coordonnées cartésiennes du point.
            \lstinputlisting[language=Caml, firstline=85, lastline=94]{code/2d.cpp}

      \item Ajoutez une fonction membre distance effectuant calculant la distance entre deux point.
            \lstinputlisting[language=Caml, firstline=68, lastline=78]{code/2d.cpp}

      \item Ajoutez une une fonction membre milieu donnant le milieu d'un segment.
            \lstinputlisting[language=Caml, firstline=94, lastline=102]{code/2d.cpp}

      \item Ecrivez un petit programme d'essai (main) gérant la classe point.
            \lstinputlisting[language=Caml, firstline=104, lastline=120]{code/2d.cpp}

\end{enumerate}
